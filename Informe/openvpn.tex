\section{OpenVPN}
Para realizar el trabajo práctico fue necesario la creación de 10 VPNs\footnote{Los 10 dispositivos son: Host A, Host B, Host C, DNS1, DNS2, DNS Root, WebServer, FTPServer y Telserver, que cuenta con dos interfaces tap}, para poder conectar los distintos dispositivos físicos a la red simulada en GNS3. Notamos que, aunque sólo 5 de éstos servicios estarían en simultáneo utilizándose (los DNSs, alguno de los 3 hosts, y alguno de los servicios), es necesario tener todo configurado para poder cambiar a otro dispositivo (siendo mucho más fácil configurando distintas VPNs).

\subsection{Servidores}
Dentro de la PC que corre la topología en GNS3 se tiene a los 9 VPNs ejecutándose en paralelo (en 9 puertos diferentes). Para cada uno de estos se configura la interfaz tap correspondiente. 

\subsection{Clientes}
Para cada uno de los clientes se crearon los scripts correspondientes para poder conectar al servidor, el cual cuenta con la dirección IP con la que deberá conectarse, el puerto al que debe conectarse (para conectarse a una determinada interfaz tap del servidor), el certificado, máscara de subred, y otros parámetros. Además, es necesario configurar la tabla de ruteo para tal cliente (como ya fue determinada en la sección de Tablas de Ruteo Estático).
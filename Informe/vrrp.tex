\section{VRRP}
Para la realización del trabajo práctico es necesario implementar VRRP en un determinado par de routers para permitir la redundancia (en nuestro caso, entre los routers R4 y R5). Para esto, también es necesario configurar el Object Tracking correspondiente, para poder conocer el estado del otro router (ya sea si el router o alguno de sus enlaces se ha caido).
\subsection{Configuración de los Routers}
\paragraph{R4}
{\small
\begin{verbatim}
track 1 interface Ethernet0/0 ip routing
track 2 interface Ethernet0/1 ip routing

interface Ethernet0/0
 description Conexion LAN Red A
 ip address 201.158.15.3 255.255.255.128
 full-duplex
 vrrp 10 ip 201.158.15.7
 vrrp 10 timers advertise 15
 vrrp 10 timers learn
 vrrp 10 priority 110
 vrrp 10 track 1 decrement 20
 vrrp 10 track 2 decrement 20

interface Ethernet0/1
 description Conexion LAN Red D
 ip address 20.64.73.2 255.255.255.0
 full-duplex
 vrrp 11 ip 20.64.73.5
 vrrp 11 timers advertise 15
 vrrp 11 timers learn
 vrrp 11 priority 110
 vrrp 11 track 1 decrement 20
 vrrp 11 track 2 decrement 20
\end{verbatim}
}

\paragraph{R5}
{\small
\begin{verbatim}
track 1 interface Ethernet0/0 ip routing
track 2 interface Ethernet0/1 ip routing

interface Ethernet0/0
 description Conexion LAN Red A
 ip address 201.158.15.4 255.255.255.128
 full-duplex
 vrrp 10 ip 201.158.15.7
 vrrp 10 timers advertise 15
 vrrp 10 timers learn
 vrrp 10 priority 100
 vrrp 10 track 1 decrement 20
 vrrp 10 track 2 decrement 20

interface Ethernet0/1
 description Conexion LAN Red D
 ip address 20.64.73.1 255.255.255.0
 full-duplex
 vrrp 11 ip 20.64.73.5
 vrrp 11 timers advertise 15
 vrrp 11 timers learn
 vrrp 11 priority 100
 vrrp 11 track 1 decrement 20
 vrrp 11 track 2 decrement 20
\end{verbatim}
}

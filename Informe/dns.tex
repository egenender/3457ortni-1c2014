\section{DNS}

\subsubsection{Niveles de servidores DNS}

Para la realización del trabajo práctico es necesario implementar el servicio de DNS, dividido en las distintas sedes de la topología, para así poder traducir nombres a direcciones IP (para permitir la comunicación) o viceversa (por ejemplo, para el uso de traceroute). \\ 

Se dispone de tres servidores DNS, uno de primer nivel (DNS root) y dos de segundo nivel (llamados DNS 1 y DNS 2). El DNS 1 tiene autoridad sobre la sede La Falda y el DNS 2 sobre el resto de la red (sedes Calamuchita y Córdoba Capital). Los hosts que deseen traducir un nombre o una IP deben consultar al servidor DNS de segundo nivel que corresponda a su zona. En caso de que dicho servidor pueda realizar la traducción, transmite la consulta hacia el DNS Root. \\

\subsubsection{Nombres de dominio de redes}


\begin{tabular}{|c|c|c|c|c|}
	\hline
	Subnet & Nombre & Nombre de dominio &  Sede\\
	\hline
	A & Águila & aguila.capital.cordoba.dc.fi.uba.ar &  Capital \\
	\hline
	B & Buitre & buitre.capital.cordoba.dc.fi.uba.ar & Capital \\
	\hline
	C & Cuervo & cuervo.capital.cordoba.dc.fi.uba.ar & Capital \\
	\hline
	D & Dodo & dodo.capital.cordoba.dc.fi.uba.ar & Capital \\
	\hline
	E & Espátula & espatula.lafalda.cordoba.dc.fi.uba.ar & La Falda \\
	\hline
	F & Faisán & faisan.lafalda.cordoba.dc.fi.uba.ar & La Falda \\
	\hline
	G & Golondrina & golondrina.calamuchita.cordoba.dc.fi.uba.ar & Calamuchita \\
	\hline
	H & Hornero & hornero.calamuchita.cordoba.dc.fi.uba.ar & Calamuchita \\
	\hline
	I1 & Ibis 1 & ibis1.calamuchita.cordoba.dc.fi.uba.ar & Calamuchita \\
	\hline
	I2 & Ibis 2 & ibis2.calamuchita.cordoba.dc.fi.uba.ar &  Calamuchita \\ 
	\hline
	I3 & Ibis 3 & ibis3.calamuchita.cordoba.dc.fi.uba.ar &  Calamuchita \\
	\hline
	I4 & Ibis 4 & ibis4.calamuchita.cordoba.dc.fi.uba.ar & Calamuchita \\
	\hline
	I5 & Ibis 5 & ibis5.calamuchita.cordoba.dc.fi.uba.ar &  Calamuchita \\	
	\hline
	I6 & Ibis 6 & ibis6.calamuchita.cordoba.dc.fi.uba.ar &  Calamuchita \\
	\hline
	I7 & Ibis 7 & ibis7.calamuchita.cordoba.dc.fi.uba.ar &  Calamuchita \\
	\hline
	I8 & Ibis 8 & ibis8.calamuchita.cordoba.dc.fi.uba.ar &  Calamuchita \\
	\hline
	I9 & Ibis 9 & ibis9.calamuchita.cordoba.dc.fi.uba.ar &  Calamuchita \\
	\hline
	I10 & Ibis 10 & ibis10.calamuchita.cordoba.dc.fi.uba.ar &  Calamuchita \\
	\hline
	J & Jilguero & jilguero.lafalda.cordoba.dc.fi.uba.ar &  La Falda \\
	\hline
	K & Kiwi & kiwi.lafalda.cordoba.dc.fi.uba.ar & La Falda \\
	\hline
	L & Lechuza & lechuza.calamuchita.cordoba.dc.fi.uba.ar & Calamuchita \\
	\hline
	M1 & Mero 1 & mero1.calamuchita.cordoba.dc.fi.uba.ar & Calamuchita \\
	\hline
	M2 & Mero 2 & mero2.calamuchita.cordoba.dc.fi.uba.ar & Calamuchita \\
	\hline
	N privada & Negrón & negron.calamuchita.cordoba.dc.fi.uba.ar & Calamuchita \\
	\hline
	O privada & Ortega & ortega.calamuchita.cordoba.dc.fi.uba.ar & Calamuchita \\
	\hline
	P privada & Paloma & paloma.calamuchita.cordoba.dc.fi.uba.ar & Calamuchita \\
	\hline
	Q1 Publica & Quebrantahuesos 1 & quebrantahuesos1.calamuchita.cordoba.dc... &  Calamuchita \\
	\hline
	Q2 Publica & Quebrantahuesos 2 & quebrantahuesos2.calamuchita.cordoba.dc... & Calamuchita \\
	\hline
	Q3 Publica & Quebrantahuesos 3 & quebrantahuesos3.calamuchita.cordoba.dc... &  Calamuchita \\
	\hline
\end{tabular}


